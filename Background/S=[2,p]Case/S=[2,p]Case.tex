\documentclass[11pt]{article}

% relevant packages
\usepackage[parfill]{parskip} 	% Activate to begin paragraphs with an empty line rather than an indent
\usepackage{graphicx}
\usepackage{amssymb}
\usepackage{amsmath}
\usepackage{mathrsfs }
\usepackage{amsthm}
\usepackage{epstopdf}
\usepackage{enumerate}
\usepackage{tikz}
\usetikzlibrary{matrix}
\usepackage{listings}
\usepackage{color}
\usepackage[all]{xy}
\usepackage[english]{babel}
\usepackage{setspace}

\DeclareGraphicsRule{.tif}{png}{.png}{`convert #1 `dirname #1`/`basename #1 .tif`.png}

% stylistic environment
\definecolor{grau}{rgb}{0.3,0.3,0.3}
\usepackage[colorlinks, linkcolor=grau, citecolor=grau, urlcolor=grau]{hyperref}
\usepackage{breakurl}
\urlstyle{same}
\usepackage[top=1.3in, bottom=1.6in, left=1.3in, right=1.3in]{geometry}
\frenchspacing
\sloppy
\usepackage{booktabs}
\usepackage[ruled,vlined]{algorithm2e}
\usepackage{enumerate}
\onehalfspacing

% relevant environments
\newtheorem{theorem}{Theorem}[section]
\newtheorem{conjecture}[theorem]{Conjecture}
\newtheorem{corollary}[theorem]{Corollary}
\newtheorem{lemma}[theorem]{Lemma}
\newtheorem{properties}[theorem]{Properties}
\newtheorem{proposition}[theorem]{Proposition}
\newtheorem{problem}[theorem]{Problem}
\newtheorem{question}[theorem]{Question}

\theoremstyle{definition}
\newtheorem{Algorithm}[theorem]{Algorithm}
\newtheorem{definition}[theorem]{Definition}
\newtheorem{example}[theorem]{Example}
\newtheorem{remark}[theorem]{Remark}

% math operators
\DeclareMathOperator{\ord}{ord}
\DeclareMathOperator{\sgn}{sgn}
\DeclareMathOperator{\rad}{rad}

% mathbb characters
\newcommand{\QQ}{\mathbb{Q}}
\newcommand{\ZZ}{\mathbb{Z}}
\newcommand{\CC}{\mathbb{C}}

% code environment
\definecolor{dkgreen}{rgb}{0,0.6,0}
\definecolor{gray}{rgb}{0.5,0.5,0.5}
\definecolor{mauve}{rgb}{0.58,0,0.82}

\lstset{%frame=tb,
  language=Java,
  aboveskip=3mm,
  belowskip=3mm,
  showstringspaces=false,
  columns=flexible,
  basicstyle={\small\ttfamily},
  numbers=none,
  numberstyle=\tiny\color{gray},
  keywordstyle=\color{blue},
  commentstyle=\color{dkgreen},
  stringstyle=\color{mauve},
  breaklines=true,
  breakatwhitespace=true,
  tabsize=3
}

%----------------------------------------------------------------------------------------------------------------------------------------------%
%----------------------------------------------------------------------------------------------------------------------------------------------%

\title{The Sum of Two $S$-Units Being a Square \\ 
	Where $S = \{2,p\}$}
	
\author{}
%\date{}                                           % Activate to display a given date or no date

\begin{document}
\maketitle

%----------------------------------------------------------------------------------------------------------------------------------------------%
%----------------------------------------------------------------------------------------------------------------------------------------------%

Suppose $S = \{2,p\}$, where $p \geq 3$ is a prime and let $\tilde{S}$ denote the set of positive rational integers which have no prime divisors outside of $S$. We study the equation
\[x + y = z^2\]
in $x,y$ $\tilde{S}$-units, and $z \in \mathbb{Q}$, where the set of $\tilde{S}$-unit in this case is defined as
\[\{ \pm 2^{x_1}p^{x_2} \ | \ x_i \in \mathbb{Z} \text{ for } i = 1, 2\}.\]

Clearing denominators, without loss of generality, we may study the equation 
\[x + y = z^2\]
where 
\[\begin{cases}
x \in \tilde{S}, & \pm y \in \tilde{S}, \\
x \geq y, & z \in \mathbb{Z}, \\
z > 0, & \gcd(x,y) \text{ is squarefree}.
\end{cases}\]

In other words, 
\[\begin{cases}
x \geq 0, & \rad(x) | 2p, \\
y \in \mathbb{Z}, & \rad(y) | 2p, \\
x \geq y, & z \in \mathbb{Z}, \\
z > 0, & \gcd(x,y) \text{ is squarefree}.
\end{cases}\]

Hence let $x = 2^{a}p^b$ and $y = 2^{c}p^d$. Since
\[ \gcd(x,y) = \gcd(2^ap^b, 2^bp^d) = 2^{\min(a,c)}p^{\min(b,d)}\]
is necessarily squarefree, it follows that $\min(a,c) \leq 1$ and $\min(b,d) \leq 1$. This leaves us several cases to consider. 

\begin{center}
\begin{tabular}{| c | r | c |} \hline
Case	 	& $x + y = z^2$ 			&  Additional Conditions \\ \hline \hline
		& $2^a + p^d = z^2$			& \\
1.		& $- 2^a + p^d = z^2$ 		&  $a \geq 0$, $d \geq 0$ \\ 
		& $2^a - p^d = z^2$			& \\ \hline
		& $2^ap + p^d = z^2$		& \\
2.		& $- 2^ap + p^d = z^2$	 	&  $a \geq 0$, $d \geq 1$ \\
		& $2^ap - p^d = z^2$			& \\ \hline
		& $2^ap^b + 1 = z^2$		& \\
3.		& $-2^ap^b + 1 = z^2$ 		&  $a \geq 0$, $b \geq 0$ \\
		& $2^ap^b - 1 = z^2$			& \\ \hline
		& $2^ap^b + p = z^2$		& \\
4.		& $-2^ap^b + p = z^2$ 		&  $a \geq 0$, $b \geq 1$ \\
		& $2^ap^b - p = z^2$			& \\ \hline
		& $2^a + 2p^d = z^2$		& \\
5.		& $-2^a + 2p^d = z^2$	 	&  $a \geq 1$, $d \geq 0$ \\ 
		& $2^a - 2p^d = z^2$			& \\ \hline
		& $2^ap + 2p^d = z^2$		& \\
6.		& $-2^ap + 2p^d = z^2$ 		&  $a \geq 1$, $d \geq 1$ \\
		& $2^ap - 2p^d = z^2$		& \\ \hline
		& $2^ap^b + 2 = z^2$		& \\
7.		& $-2^ap^b + 2 = z^2$ 		&  $a \geq 1$, $b \geq 0$ \\ 
		& $2^ap^b - 2 = z^2$			& \\ \hline
		& $2^ap^b + 2p = z^2$		& \\ 
8.		& $-2^ap^b + 2p = z^2$ 		&  $a \geq 1$, $b \geq 1$ \\ 
		& $2^ap^b - 2p = z^2$		& \\ \hline
\end{tabular}
\end{center}




















\end{document}
