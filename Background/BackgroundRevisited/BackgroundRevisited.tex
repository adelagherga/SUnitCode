\documentclass[11pt]{article}

% relevant packages
\usepackage[parfill]{parskip} 	% Activate to begin paragraphs with an empty line rather than an indent
\usepackage{graphicx}
\usepackage{amssymb}
\usepackage{amsmath}
\usepackage{mathrsfs }
\usepackage{amsthm}
\usepackage{epstopdf}
\usepackage{enumerate}
\usepackage{tikz}
\usetikzlibrary{matrix}
\usepackage{listings}
\usepackage{color}
\usepackage[all]{xy}
\usepackage[english]{babel}
\usepackage{setspace}

\DeclareGraphicsRule{.tif}{png}{.png}{`convert #1 `dirname #1`/`basename #1 .tif`.png}

% stylistic environment
\definecolor{grau}{rgb}{0.3,0.3,0.3}
\usepackage[colorlinks, linkcolor=grau, citecolor=grau, urlcolor=grau]{hyperref}
\usepackage{breakurl}
\urlstyle{same}
\usepackage[top=1.3in, bottom=1.6in, left=1.3in, right=1.3in]{geometry}
\frenchspacing
\sloppy
\usepackage{booktabs}
\usepackage[ruled,vlined]{algorithm2e}
\usepackage{enumerate}
\onehalfspacing

% relevant environments
\newtheorem{theorem}{Theorem}[section]
\newtheorem{conjecture}[theorem]{Conjecture}
\newtheorem{corollary}[theorem]{Corollary}
\newtheorem{lemma}[theorem]{Lemma}
\newtheorem{properties}[theorem]{Properties}
\newtheorem{proposition}[theorem]{Proposition}
\newtheorem{problem}[theorem]{Problem}
\newtheorem{question}[theorem]{Question}

\theoremstyle{definition}
\newtheorem{Algorithm}[theorem]{Algorithm}
\newtheorem{definition}[theorem]{Definition}
\newtheorem{example}[theorem]{Example}
\newtheorem{remark}[theorem]{Remark}

% math operators
\DeclareMathOperator{\ord}{ord}
\DeclareMathOperator{\rad}{rad}
\DeclareMathOperator{\sgn}{sgn}

% mathbb characters
\newcommand{\QQ}{\mathbb{Q}}
\newcommand{\ZZ}{\mathbb{Z}}
\newcommand{\CC}{\mathbb{C}}

% code environment
\definecolor{dkgreen}{rgb}{0,0.6,0}
\definecolor{gray}{rgb}{0.5,0.5,0.5}
\definecolor{mauve}{rgb}{0.58,0,0.82}

\lstset{%frame=tb,
  language=Java,
  aboveskip=3mm,
  belowskip=3mm,
  showstringspaces=false,
  columns=flexible,
  basicstyle={\small\ttfamily},
  numbers=none,
  numberstyle=\tiny\color{gray},
  keywordstyle=\color{blue},
  commentstyle=\color{dkgreen},
  stringstyle=\color{mauve},
  breaklines=true,
  breakatwhitespace=true,
  tabsize=3
}

%----------------------------------------------------------------------------------------------------------------------------------------------%
%----------------------------------------------------------------------------------------------------------------------------------------------%

\title{Algorithms for $S$-Unit Equations, Revisited}
	
\author{}
%\date{}                                           % Activate to display a given date or no date

\begin{document}
\maketitle

%----------------------------------------------------------------------------------------------------------------------------------------------%
%----------------------------------------------------------------------------------------------------------------------------------------------%

\section{de Weger's Approach}

Let $S = \{p_1, \dots, p_s\}$ be a finite set of rational primes, write $N_S = \prod_{p \in S} p$ and denote by $\mathcal{O}^{\times}$ the group of units of $\mathcal{O} = \mathbb{Z}[1/N_S]$. We are interested in solving the $S$-unit equation
\begin{equation} \label{Eq:1}
x + y = 1, \quad (x,y) \in \mathcal{O}^{\times} \times \mathcal{O}^\times.
\end{equation}

Suppose that $(x,y)$ satisfies \eqref{Eq:1}. Then there exists nonzero $a,b,c \in \mathbb{Z}$ with $\gcd(a,b,c) = 1$ and $\rad(abc) \mid N_S$, such that $x = \frac{a}{c}$, $y = \frac{b}{c}$, and $a + b = c$. Indeed, since $(x,y) \in \mathcal{O}^{\times} \times \mathcal{O}^\times$, we have $x = \frac{\alpha_1}{\beta_1}, y = \frac{\alpha_2}{\beta_2}$, where $(\alpha_1, \beta_1) = 1$, $(\alpha_2 \beta_2) = 1$, and $\alpha_1, \beta_1, \alpha_2, \beta_2$ are integers composed only of primes of $S$. Then
\[x + y = \frac{\alpha_1}{\beta_1} + \frac{\alpha_2}{\beta_2} = 1\]
yields 
\[\alpha_1 \beta_2 + \alpha_2 \beta_1 = \beta_1 \beta_2\]
and we take $a = \alpha_1 \beta_2$, $b = \alpha_2 \beta_1$ and $c = \beta_1 \beta_2$. Clearly, $\rad(abc) \mid N_S$. We note lastly that if a prime $p \in S$ divides two of $a,b,c$, then it must also divide the third integer, so that 









































\end{document}